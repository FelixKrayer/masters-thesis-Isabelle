% !TeX root = ../main.tex
% Add the above to each chapter to make compiling the PDF easier in some editors.

\chapter{Background}
\label{chapter:background}

  This Chapter serves to introduce \acp{BNF} and their generalization to \acp{MRBNF}. 
  \section{Bounded Natural Functors (BNFs)}
    As described in \autoref{chapter:introduction}, \acp{BNF} are essential for constructing datatypes and codatatypes in Isabelle/HOL. Especially defining a recursive datatype like 
    \begin{equation*}
      \textbf{datatype}\; 'a\; \textsf{ex} = \text{A}\; "('a\; \times\; 'a\; \textsf{ex})\; \textsf{list}"
    \end{equation*}

    it is required that the type constructor \textsf{$'a$ list} is registered as a \ac{BNF}, i.e., the \ac{BNF}-axioms have been shown for it. Non-\ac{BNF} types like \textsf{$'a$ set} may be used in a \textbf{datatype} command, but they cannot be used to recurse. Since \acp{BNF} are closed under composition and fixpoints, the resulting datatype (in the example \textsf{$'a$ ex}) can be automatically registered as a \ac{BNF} as well.

    Another example of a \ac{BNF} is the product type \textsf{($'a$, $'b$) prod} with infix notation \textsf{$'a\; \times\; 'b$}. This is a binary type constructors.
    
    Furthermore, \textsf{$'a$ list} and \textsf{($'a$, $'b$) prod} have a well-defined set and map function. The exact properties are listed below. Thus they behave \textit{natural}. Lastly, they are \textit{bounded}, since the set of elements that the type can describe is bounded by a (possibly transfinite) cardinal.

    \begin{align}
      \text{\textit{map\_id}: }& \text{\textsf{map}}_F\; id\; x = x\\
      \text{\textit{map\_comp}: }& \text{\textsf{map}}_F\; g\; (\text{\textsf{map}}_F\; f\; x) = \text{\textsf{map}}_F\; (g \circ f)\; x\\
      \text{\textit{map\_cong}: }& (\forall z \in \text{\textsf{set}}_F\; z.\; f\; z = g\; z) \Longrightarrow 
        \text{\textsf{map}}_F\; f\; x = \text{\textsf{map}}_F\; g\; x\\
      \text{\textit{set\_map}: }& \text{\textsf{set}}_F (\text{\textsf{map}}_F\; f\; x) = f\; \grave{}\; \text{\textsf{set}}_F\; x\\
      \text{\textit{infinite\_regular\_card\_order}: }& \text{\textsf{infinite}}\; \text{\textsf{bound}}_F \land 
        \text{\textsf{regular}}\; \text{\textsf{bound}}_F \land 
        \text{\textsf{cardinal\_order}}\; \text{\textsf{bound}}_F\\
      \text{\textit{set\_bd}: }& | \text{\textsf{set}}_F x | <_o \text{\textsf{bound}}_F\\
      \text{\textit{rel\_compp\_leq}: }& \text{\textsf{rel}}_F\; R\; \bullet\; \text{\textsf{rel}}_F\; Q\; = 
        \text{\textsf{rel}}_F\; (R\; \bullet\; Q)\\
      \text{\textit{in\_rel}: }& \text{Weak Pullback Preservation WP}
    \end{align}
    Where $\grave{}$ is the image function on sets and $\bullet$ is the composition of relations. Furthermore $<_o$ is the less than relation on the level of cardinals

    While most of these properties are straightforward, we want to explain the preservation of weak pullbacks in more detail.
    \begin{equation}
      \text{\textsf{rel}}_F\; R\; x\; y = \exists z.\; \text{\textsf{set}}_F\; z \subseteq \{(a, b).\; R\; a\; b \} \land 
        \text{\textsf{map}}_F\; fst\; z = x \land \text{\textsf{map}}_F\; snd\; z = y \label{eq:WP}
    \end{equation}
    The idea is that two elements $x$ and $y$ of the type $\alpha\; F$ are related through a relation $R$ iff there exists a $z$ that acts as a "zipped" version of $x$ and $y$. The atoms of this $z$ are the atoms of $x$ and $y$, that are organized in pairs of $R$-related with the $x$ as the first and $y$ as the second position in the pair.

  \section{Map-Restricted Bounded Natural Functors (MRBNFs)}
    \acp{MRBNF} are a generalization of \acp{BNF}. Restricting the \textsf{map} function of a functor to \textit{small-support} functions or \textit{small-support bijections} for certain type variables allows us to reason about type constructors in terms of \ac{BNF} properties, even in cases where this would not be possible otherwise. We call type variables that that are restricted to small-support functions \textit{free} variables  and those restricted to small-support bijections \textit{bound} variables. This allows us to define \acp{MRBNF} with four types of variables (live, free, bound and dead) as opposed to \acp{BNF} which only distinguish between lives and deads. 
    Consequently, for type constructors with variables that are considered dead in \ac{BNF} terms, we can declare some of them as free or bound variables, depending on the type.

    
  \begin{itemize}
    \item Explain all the BNF theorems
    \item cite \cite{blanchette2019bindings}
  \end{itemize}

