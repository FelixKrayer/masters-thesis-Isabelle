% !TeX root = ../main.tex
% Add the above to each chapter to make compiling the PDF easier in some editors.

\chapter{Background}
\label{chapter:background}

  This Chapter serves to introduce \acp{BNF} and their generalization to \acp{MRBNF}. 
  \section{Bounded Natural Functors (BNFs)}
    As described in \autoref{chapter:introduction}, \acp{BNF} are essential for constructing datatypes and codatatypes in Isabelle/HOL. Especially for defining a datatype with recursion it is required that the type constructor \textsf{$'a$ list} is registered as a \ac{BNF}, i.e., it fulfills the \ac{BNF}-axioms. For example the following \textbf{datatype} command only succeeds if \textsf{$'a$ list}, is a \ac{BNF}.
    \begin{equation*}
      \textbf{datatype}\; 'a\; \textsf{ex} = \text{A}\; "('a\; \times\; 'a\; \textsf{ex})\; \textsf{list}"
    \end{equation*}
    Since \acp{BNF} are closed under composition and fixpoints, the resulting datatype (here \textsf{$'a$ ex}) can be automatically registered as a \ac{BNF} as well.


    The type variables of a \ac{BNF} are divided into two groups: \textit{live} and \textit{dead} variables or \textit{lives} and \textit{deads}. Live variables can be used for recursive definitions, while dead ones do not allow for this. 
    We list a few important \acp{BNF} and note the liveness of their type variables:\\

    \begin{tabular}{l|l}
      \textsf{$'a$ list} & $'a$ is live\\
      \textsf{($'a$, $'b$) prod} (infix notation: \textsf{$'a\; \times\; 'b$}) & $'a$ and $'b$ are live\\
      \textsf{($'a$, $'b$) fun} (infix notation: \textsf{$'a\; \Rightarrow\; 'b$}) & $'a$ is dead, $'b$ is live\\
    \end{tabular}\\

    A \ac{BNF} $F$ with $n$ live variables is characterized by map and set functions, a relator and a bound. The map function or \textit{mapper} takes functions at the level of $F$'s lives as arguments, i.e., the domains of these functions are the lives of $F$. These functions are recursively applied to the atoms of an element of the type. The result is a new element of type $F$, where the original type variables have been replaced by the range types of the mapped functions.

    Additionally, a set function or \textit{setter} exists for each live variable. Applied to an $F$-element, the $i$-th setter returns the set of all atoms that are part of the element in some place.

    % relator, bound
    % example list/prod

    \subsection{BNF-axioms}
      A \ac{BNF} is characterized by map and set functions, a relator and a bound. We consider a \ac{BNF} $F$ with $n$ live variables and use the notation $\overline{f} = f_1 \dots f_n$. $\dot{G}$. Furthermore, when we write $i$ as an index, we assume it to be in the range $1 \leq i \leq n$.
      We give the \ac{BNF}-axioms as follows:
      \begin{align}
        \textsc{map\_id: }& \textsf{map}_F\; \overline{id}\; x = x\\
        \textsc{map\_comp: }& \textsf{map}_F\; \overline{g}\; (\textsf{map}_F\; \overline{f}\; x) = \textsf{map}_F\; \overline{(g \circ f)}\; x\\
        \textsc{map\_cong: }& (\forall i.\; \forall z \in \textsf{set}_{F,i}\; x.\; f_i\; z = g_i\; z) \Longrightarrow 
          \textsf{map}_F\; \overline{f}\; x = \textsf{map}_F\; \overline{g}\; x\\
        \textsc{set\_map: }& \forall i.\; \textsf{set}_{F,i} (\textsf{map}_F\; \overline{f}\; x) = f_i\; \grave{\phantom{\_}}\; \textsf{set}_{F,i}\; x\\
        \textsc{bd: }& \textsf{infinite}\; \textsf{bd}_F \land 
          \textsf{regular}\; \textsf{bd}_F \land 
          \textsf{cardinal\_order}\; \textsf{bd}_F\\
        \textsc{set\_bd: }& \forall i.\; | \textsf{set}_{F,i}\; x | <_o \textsf{bd}_F\\
        \textsc{rel\_compp\_leq: }& \textsf{rel}_F\; \overline{R}\; \bullet\; \textsf{rel}_F\; \overline{Q}\; = 
          \textsf{rel}_F\; \overline{(R\; \bullet\; Q)}\\
        \textsc{in\_rel: }& \text{Weak Pullback Preservation WP}
      \end{align}
      \noindent where $\grave{\phantom{\_}}$ is the image function on sets and $\bullet$ is the composition of relations. Furthermore $<_o$ is the less than relation on the level of cardinals
      % Explain the other axioms

      While most of these properties are straightforward, we want to explain the preservation of weak pullbacks in more detail.
      \begin{equation}
        \textsf{rel}_F\; \overline{R}\; x\; y = 
          \exists z.\; (\forall i.\; \textsf{set}_{F,i}\; z \subseteq \{(a, b).\; R_i\; a\; b \}) \land 
          \textsf{map}_F\; \overline{fst}\; z = x \land \textsf{map}_F\; \overline{snd}\; z = y \label{eq:WP}
      \end{equation}
      The idea is that two elements $x$ and $y$ of the type $\alpha\; F$ are related through a relation $R$ iff there exists a $z$ that acts as a "zipped" version of $x$ and $y$. The atoms of this $z$ are the atoms of $x$ and $y$, that are organized in pairs of $R$-related with the $x$ as the first and $y$ as the second position in the pair.

    \subsection{BNF examples}
      Further examples of \acp{BNF} are is the product type \textsf{($'a$, $'b$) prod}, a binary type constructor with infix notation \textsf{$'a\; \times\; 'b$}, and the type of finite sets \textsf{$'a$ fset}. The latter is interesting for the reason that it is a subtype of the set type, which is not a \ac{BNF}. By enforcing finiteness for the elements of the type it is possible to give a bound for the set function, fulfilling the \textsc{set\_bd} axiom, which is not possible for the unrestricted set type. Since unboundedness is the only reason that the set type is not a \ac{BNF}, \textsf{$'a$ fset} can be shown to be a \ac{BNF}.
      
      To show, how \acp{BNF} can be combined to create new ones, we consider the type constructor \textsf{($'a$, $'b$) plist} = \textsf{($'a\; \times\; 'b)$ list}. We define for it a map function ($\textsf{map}_\textsf{plist}$) and two set functions ($\textsf{set1}_\textsf{plist}$ and $\textsf{set2}_\textsf{plist}$) as well as a relator $\textsf{rel}_\textsf{plist}\; R\; S$. The exact definitions are given as such:
      \begin{align*}
        \textsf{map}_\textsf{plist}\; f\; g\; &= \textsf{map}_\textsf{list}\; (\textsf{map}_\textsf{prod}\; f\; g)\\
        \textsf{set1}_\textsf{plist}\; xs &= \textsf{set}_\textsf{list}\; (\textsf{map}_\textsf{list}\; fst\; xs)\\
        \textsf{set2}_\textsf{plist}\; xs &= \textsf{set}_\textsf{list}\; (\textsf{map}_\textsf{list}\; snd\; xs)\\
        \textsf{rel}_\textsf{plist}\; R\; S &= \textsf{rel}_\textsf{list} (\textsf{rel}_\textsf{prod}\; R\; S)
      \end{align*}
      \noindent where we use the standard map, set and relator functions of the list and product type.

      To show that \textsf{($'a$, $'b$) plist} is a \ac{BNF}, we have to prove the \ac{BNF}-axioms for it. Besides the definitions above, a bound $\textsf{bd}_\textsf{plist}$ is needed. We chose $natLeq$. 
      
      % TODO: EXPLAIN bd and natLeq

  \section{Subtype}
    We can carve out a subtype from a type constructor using the \textbf{typedef} command. 
    % The idea was to have this here to show that there exist BNFs that have a dead variable, that are not dead in an MRBNF (but bound). Maybe this idea has to go to the 3 chapter 

  \section{Map-Restricted Bounded Natural Functors (MRBNFs)}
    \acp{MRBNF} are a generalization of \acp{BNF}. Restricting the \textsf{map} function of a functor to \textit{small-support} functions or \textit{small-support bijections} for certain type variables allows us to reason about type constructors in terms of \ac{BNF} properties, even in cases where this would not be possible otherwise. We call type variables that that are restricted to small-support functions \textit{free} variables  and those restricted to small-support bijections \textit{bound} variables. This allows us to define \acp{MRBNF} with four types of variables (live, free, bound and dead) as opposed to \acp{BNF} which only distinguish between lives and deads. 

    \acp{MRBNF} can be used in a \textbf{binder\_datatype} command to produce a datatype with bindings. 
    % binder_datatype command
    
    Consequently, for type constructors with variables that are considered dead in \ac{BNF} terms, we can declare some of them as free or bound variables, depending on the type.

    
  \begin{itemize}
    \item cite \cite{blanchette2019bindings}
  \end{itemize}

