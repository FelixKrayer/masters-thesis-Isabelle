% !TeX root = ../main.tex
% Add the above to each chapter to make compiling the PDF easier in some editors.

\chapter{Introduction}
\label{chapter:introduction}

  Isabelle/HOL provides a fleshed out system for defining datatypes and codatatypes through the \textbf{datatype} command. This system is built on the foundation of \acp{BNF}.

  This allows to express and thus reason about many different kinds of structured concepts. (Induction!)

  How about expressing simple lambda terms? For example\dots

  While lambda terms could be expressed as\dots, this has a number of problems. Mainly it allows for many non-valid terms and does not consider alpha-equivalence.

  It is possible to carve out subtypes with Isabelle's \textbf{typedef} command (and quotient?), however thereby the \ac{BNF}-Properties are lost.

  Recently a new way of representing Binding aware datatypes was presented in the \textbf{binder\_datatype} command that is build on the theory of \acp{MRBNF}

  While this solves some issues, it relies on manually linearizing types, i.e. creating types that are non-repettitive. These are not closed under \dots and thus they cannot be structurally defined. Instead a non-linear type is built and then they are carved out. This was a manual process but now we automate it.

  - Datatypes in Isabelle/HOL are built on \acp{BNF} (defined in \cite{traytel2012foundational})

  - Difficult represent lambda terms (alpha equivalence, capture-avoiding substitution) ==> binding-aware datatypes

  - Structure of the Thesis
