% !TeX root = ../main.tex
% Add the above to each chapter to make compiling the PDF easier in some editors.

\chapter{Linearizing MRBNFs}
\label{chapter:methodology}

  \section{Linearization of MRBNFs (In theory)}

  In this section we define the linearization of an MRBNF on a subset of it's \textit{live} variables. The result of the linearization is a new MRBNF with the same variable types (\textit{live}, \textit{dead}, \textit{bound}, \textit{free}), except for the linearized variables that change their type from \textit{live} to \textit{bound}. This means that the map function is now restricted to only allow bijective and small-support functions on these variables. Apart from this change, it is ensured that the MRBNF is \textit{nonrepetitive} with respect to the linearized variables. We give a definition nonrepetitiveness in the following \autoref{subsec:nonrep}. Intuitively it means that the atoms of that type cannot occur multiple times in an element of the type.

    \subsection{Nonrepetitiveness}
    \label{subsec:nonrep}
    At the core of linearization lies the notion of \textit{nonrepetitiveness}. We think of an element $x$ of a type $\alpha\; F$ as being nonrepetitive if all its $\alpha$ atoms are distinct from another. We give an exact definition \textsf{nonrep} of nonrepetitiveness in relation to all other elements of $\alpha\; F$ with the \textit{same shape}:

    \begin{align}
      \text{\textsf{same shape}}_F\; x\; y &= \text{\textsf{rel}}_F\; top\; x\; y\\
      \text{\textsf{nonrep}}_F\; x &= \forall y.\; \text{\textsf{same shape}}\; x\; y \longrightarrow (\exists f.\; y = \text{\textsf{map}}_F\; f\; x)
    \end{align}

    We consider two $\alpha\; F$ elements $x$ and $y$ to have the \textit{same shape}, if they are related with the \textit{top} relation, i.e., the relation that relates all $\alpha$ atoms to all others. This is true, if the relator \textsf{rel}$_F$ can relate the elements. In the case of \textsf{list}, this is the case when two lists have the same length but possibly different content. 

    Based on this, $x$ is a nonrepetitive element, if for all other elements $y$ with the same shape, a function exists through which $x$ can be mapped to $y$. In our example of \textsf{list}, this holds for all lists with distinct elements (given a second list, one can easily define a function mapping the distinct elements of $x$ to that list). It does not hold for lists with repeating elements, because no $f$ exists that could map two same elements at different positions in this list to distinct elements in an arbitrary second list.

    For \acp{MRBNF} with more than one live variable, we can give a definitions of \textit{nonrepetitiveness} and having the \textit{same shape} on a subset of the live variables. In that case, we consider $x$ and $y$ of type $(\alpha, \beta)\; G$ to have the same shape with respect to $\alpha$, iff they are \textit{equal} in their $\beta$ atoms and are related with \textit{top} in $\alpha$ as before. Consequently for the map in the \textsf{nonrep} definition, the \textit{id} function is applied to the $\beta$ atoms, since they are already required to be equal. 
    \begin{align}
      \text{\textsf{same shape}}_G^1\; x\; y &= \text{\textsf{rel}}_G\; top\; (=)\; x\; y\\
      \text{\textsf{nonrep}}_G^1\; x &= \forall y.\; \text{\textsf{same shape}}\; x\; y \Longrightarrow (\exists f.\; y = \text{\textsf{map}}_G\; f\; id\; x)
    \end{align}
    
    % TODO: Explain how it is for the linearization of multiple lives?
    
  \section{Required properties}
    A \acp{MRBNF} has to fulfill two properties to be linearized. First, to ensure that the resulting type constructor is non-empty, it is required, that there exists a nonrepetitive element (with respect to the linearized variables): $\exists x.\; \text{\textsf{nonrep}}\; x$

    Furthermore, even though \acp{MRBNF} are already required to perserve weak pullbacks as defined in \autoref{eq:WP}, for the linearization it is required that they perserve all pullbacks (P). Formalized this means that the existance of $z$ in the equation has to be fulfilled uniquely, i.e., for each $R$-related $x$ and $y$ there existes \textit{exactly one} $z$ fulfilling the property in \autoref{eq:WP}. For example the full pullback preservation (P) is fulfilled by the \textsf{$\alpha$ list} and \textsf{$\alpha$ $\beta$ prod} functors but not by \textsf{$\alpha$ set}. 

  \section{Intermediate lemmas}
    \paragraph{F strong} 
      From the pullback preservation with uniqueness we can prove the following lemma. In fact this notion of strongness is equivalent to pullback preservation: 
      \begin{equation*}
        [\![\text{\textsf{rel}}_F\; R\; x\; y;\; \text{\textsf{rel}}_F\; Q\; x\; y]\!] \Longrightarrow \text{\textsf{rel}}_F\; (inf\; R\; Q)\; x\; y
      \end{equation*} 
      where the infimum $inf$ of two relations $R$ and $Q$ relates all elements that are related by both $R$ and $Q$.

    \paragraph{rel exchange}
      TODO: only interesting for functors with more than one variable
      \begin{equation*}
        [\![\text{\textsf{rel}}_G\; R_1\; R_2\; x\; y;\; \text{\textsf{rel}}_G\; Q_1\; Q_2\; x\; y]\!] \Longrightarrow \text{\textsf{rel}}_G\; R_1\; Q_2\; x\; y
      \end{equation*} 

    \paragraph{map peresrving nonrepetitiveness}
      \begin{equation*}
        [\![\text{\textsf{nonrep}}_G^1\; x;\; \text{\textsf{bijective}} f]\!] \Longrightarrow \text{\textsf{nonrep}}_G^1\; (\text{\textsf{map}}_G\; f\; g\; x)
      \end{equation*} 
      

  \section{Proving the MRBNF axioms}  

  \section{Linearization of MRBNFs (In Isabelle)}
    We implement a command that allows the user to linearize an existing \ac{MRBNF} or \ac{BNF} on one or multiple of it's live variables:
    The syntax of the command is given in the following:
    \begin{lstlisting}
linearize_mrbnf ('a :: var, 'b) lin_type = ('a :: var, 'b) type on 'a
    \end{lstlisting}

    
    % TODO: Implementation in Isabelle
    % resulting MRBNF is also "strong" (preserves all pullbacks)

  